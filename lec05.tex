\lecture{5}{29.04.2021}{}
\begin{lemma}[$\var{FOL-VALID-FIN}$ (Trakhtenbrot's Theorem)]\label{lem:Trakhtenbrot}
	$\var{FOL-VALID-FIN}$ is not recursively enumerable.
\end{lemma}
\begin{proof}
	By the following argumentation:
	\begin{align*}
		&\phi \notin \var{FOL-SAT-FIN} \\
		\iff &\mathcal{A} \not\vDash \phi \text{ for all finite structures } \mathcal{A} \\
		\iff & \mathcal{A} \vDash \neg \phi \text{ for all finite structures } \mathcal{A} \\
		\iff &\neg \phi \in \var{FOL-VALID-FIN}
	\end{align*}
	Hence, semi-decidability of $\var{FOL-VALID-FIN}$ would imply
	semi-decidability of the complement of $\var{FOL-SAT-FIN}$,
	i.e.\ decidability of $\var{FOL-SAT-FIN}$. $\contra$
\end{proof}
\newpage

For special subsets of FOL, decidability can actually be reached.
Two of these will be discussed in the following two subsections.

\subsection{Monadic first-order logic}
Monadic first-order-logic (MFO) is FOL over purely relational vocabularies,
but only monadic (unary) predicate symbols are allowed.

MFO enjoys the bounded model property, i.e.\ if a MFO formula $\phi$ is satisfiable,
then it is also satisfiable by a finite model whose domain is bounded.
\begin{lemma}[Finite model property of MFO]\label{lem:fmp of mfo}
	If $\phi$ is a satisfiable MFO-formula with $k$ predicate symbols,
	then $\phi$ has a model where the domain is a subset of $\left\{ 0,1 \right\}^k$.
\end{lemma}
\begin{proof}
	\see{43}
	The proof works analogous to $S$-filtration (see Description Logic lecture notes).
	Note that the only possibility to distinguish two elements of the domain of a given MFO-structure
	is by the atomic formulas $P_i(x)$.
	All elements that belong to exactly the same $P_i$ can therefore be identified,
	leaving only $2^k$ unique elements.
	Up to isomorphism, these are exactly the structures whose domain is a subset of $\left\{ 0,1 \right\}^k$.
\end{proof}

A decision procedure for satisfiability of a MFO-formula $\phi$ is obtained as follows:
Remember that $\phi$ is finite and therefore contains only finitely many predicate symbols and free variables.
Then observe that there are only finitely many interpretations $\mathcal{I} = (\mathcal{A}, \mathcal{V})$,
where the domain of $\mathcal{A}$ is a subset of $\left\{ 0,1 \right\}^k$.
Computing the truth value of $\phi$ under each of these $\mathcal{I}$ can be done recursively according to the definition of $\vDash$.
If $\mathcal{I} \vDash \phi$ for some $\mathcal{I}$, $\phi$ is clearly satisfiable.
In the case that $\mathcal{I} \not\vDash \phi$ for all $\mathcal{I}$, $\phi$ can not be satisfiable by lemma \ref{lem:fmp of mfo}.

\begin{corollary}[Decidability of MFO]
	The satisfiability, validity, consequence and equivalence problem are decidable for MFO.
\end{corollary}
\begin{proof}
	\begin{itemize}
		\item satisfiability: shown above
		\item validity: $\Vdash \phi \iff \neg \phi$ is unsatisfiable
		\item consequence $\phi \Vdash \psi \iff \Vdash (\phi \to \psi)$
		\item equivalence: $\phi \equiv \psi \iff \phi \Vdash \psi$ and $\psi \Vdash \phi$ 
			\qedhere
	\end{itemize}
\end{proof}
\begin{note}
	There is actually no more efficient algorithm, since the satisfiability problem for MFO is \textsc{NExpTime}-complete (without proof).
\end{note}

\newpage
\subsection{Schönfinkel-Bernays formulas}
\see{44}
A Schönfinkel-Bernays formula is a FOL-formula of the form
\[
\exists x_1, \ldots x_n \forall y_1, \ldots \forall y_m.\psi
\]
where $\psi$ is a quantifier-free formula without equality over some relational vocabulary.
For Schönfinkel-Bernays sentences an even stronger property holds, the \textit{small model property}, i.e.\ models have a size \textit{linear} in the size of the formula.
\begin{lemma}[Small model property of Schönfinkel-Bernays]
	If $\phi$ is a satisfiable Schönfinkel-Bernays formula with $n$ existential quantifiers and $l$ constant symbols,
	them $\phi$ has a model with at most $n+l$ elements.
\end{lemma}
\begin{proof}
	\see{45}
	The proof works by reducing any model $\mathcal{A}$ of $\phi$ into a substructure $B$, that only contains the elements necessary for satisfying $\phi$.
	Since $\mathcal{A} \vDash \phi$, there are some $a_1, \ldots, a_n \in \Delta^{\mathcal{A}}$ such that
	\[
		\left( \mathcal{A}, \left[ x_1 \coloneqq a_1, \ldots, x_n \coloneqq a_n \right] \right) \vDash \forall y_1 \ldots \forall y_m. \psi
	.\]
	We can then define $\mathcal{B}$ as follows:
	\begin{itemize}
		\item $\Delta^{\mathcal{B}} \defeq \left\{ a_1, \ldots, a_n \right\} \cup \left\{ c_1^{\mathcal{A}}, \ldots, c_n^{\mathcal{A}} \right\}$
		\item $c_i^{\mathcal{B}} \defeq c_i^{\mathcal{A}}$ for all constant symbols
		\item $P^{\mathcal{B}} \defeq P^{\mathcal{A}} \cap (\Delta^{\mathcal{B}})^k$ for all $k$-ary predicate symbols $P$.
	\end{itemize}
	It is then easy to see that $\left| \Delta^{\mathcal{B}} \right| \leq n + l$ and $\mathcal{B} \vDash \phi$.
\end{proof}

A decision procedure for satisfiability of Schönfinkel-Bernays formulas is obtained similarly to the decision procedure for MFO.
In this case however, only \textit{linearly} many models need to be considered.

\begin{corollary}[Decidability of Schönfinkel-Bernays]
	The satisfiability, validity, consequence and equivalence problem are decidable for Schönfinkel-Bernays formulas.
\end{corollary}

\newpage
\section{Axiomatizability of FO theories}
So far, we have looked at decidability and deductive calculi for standard FOL semantics.
In this section we will extend these considerations to formula sets that are true in some fixed structure, so called FO-theories.

\subsection{First-order theories}
\medskip
\begin{definition}[First-order theory]
	Let $\mathfrak{T}$ be a set of FOL-formulas.
	$\mathfrak{T}$ is called \textit{closed under logical consequences} if $\mathfrak{T} \Vdash \phi$ implies $\phi \in \mathfrak{T}$.
	$\mathfrak{T}$ is called an \textit{FO-theory} (or a \textit{theory}) if $\mathfrak{T}$ is true in some structure and closed under logical consequences.
\end{definition}
\begin{note}
	Each theory contains all valid formulas.
	Furthermore at most one of the formulas $\phi$ and $\neg\phi$ can belong to $\mathfrak{T}$.
	If $\mathfrak{B}$ is a satisfiable formula set that is true in some structure,
	then its logical closure $\func{Cl}(\mathfrak{B}) \defeq \left\{ \phi \mid \mathfrak{B} \Vdash \phi \right\}$ is a theory.
	In particular, $\func{Cl}(\mathfrak{B})$ is a theory for each satisfiable set of FOL-sentences.
\end{note}

\begin{definition}[Completeness of a theory]
	A theory $\mathfrak{T}$ is called \textit{complete} if for all sentences $\phi$ either $\phi \in \mathfrak{T}$ or $\neg\phi \in \mathfrak{T}$ holds.
\end{definition}
\begin{note}
	$\phi \notin \mathfrak{T}$ and $\neg \phi \notin \mathfrak{T}$ is still possible for complete theories and formulas that contain free variables.
\end{note}

\begin{lemma}\label{lem:universal closure in complete theories}
	Let $\mathfrak{T}$ be a complete theory, then the following holds for each FOL formula $\phi(\overline{x})$ :
	\[
		\phi(\overline{x}) \in \mathfrak{T} \iff \forall \overline{x}.\phi(\overline{x}) \in \mathfrak{T}
	.\]
\end{lemma}
\begin{proof}
	\see{46}
	"$ \impliedby$" holds for every theory.
	"$ \implies$" is shown using the completeness of $\mathfrak{T}$.
\end{proof}

\begin{definition}[FO-theory of a structure]
	If $\mathcal{A}$ is a structure, then the \textit{first-order theory} of $\mathcal{A}$,
	denoted by $\func{Th}(\mathcal{A})$, is defined as:
	\[
		\func{Th}(\mathcal{A}) \defeq \left\{ \phi \mid A \vDash \phi \right\}
	.\]
\end{definition}
\begin{note}
	$\func{Th}(\mathcal{A})$ is indeed a theory and obviously complete,
	because because for all sentences $\phi$, $\mathcal{A} \vDash \phi$ or $\mathcal{A} \vDash \neg \phi$,
	but never both.
\end{note}
\begin{notation}
	For finite vocabularies, FO-theories induced by structures are often written as
	\[
	\func{Th}(\Delta^{\mathcal{A}}, P_1, \ldots, P_k, f_1, \ldots, f_r)\quad\text{or}\quad\func{Th}(\Delta^{\mathcal{A}}, P_1, \ldots, P_k, f_1, \ldots, f_r, =)
	,\]
	depending on whether FOL with or without equality is used.
\end{notation}
\begin{example}
	The FO-theory of arithmetic is written as $\func{Th}(\N, +, *, =)$.
	While $<$ is not defined by the structure itself, one can easily see,
	that the following definition is indeed a correct one for this structure:
	\[
		z < y \defeq z \neq y \land \exists u.(z+u=y) 
	.\]
	For the structure of the reals $\func{Th}(\R, +, *, =)$ (also called \textit{Tarski algebra}) a different definition would be needed (because of negative numbers).
	E.g.:
	\[
		z < y \defeq z \neq y \land \exists u.(z+(u*u) = y) 
	.\]

	Consider the following sentence:
	\[
		\phi = \forall x \forall y.(x<y \to \exists z.(x<z \land z<y))
	.\]
	With the respective definition of < we get
	\[
		\phi \in \func{Th}(\R, +, *, =) \quad \text{but} \quad \phi \notin \func{Th}(\N, +, *, =)
	.\]
\end{example}
\begin{theorem}[Characterization of complete theories]
	Let $\mathfrak{T}$ be a FO-theory.
	Then, the following statements are equivalent:
	\begin{enumerate}
		\item $\mathfrak{T}$ is complete.
		\item $\mathfrak{T} = \func{Th}(\mathcal{A})$ for some structure $\mathcal{A}$.
		\item $\mathfrak{T} = \func{Th}(\mathcal{A})$ for all structures $\mathcal{A}$ such that $\mathcal{A} \vDash \mathfrak{T}$.
	\end{enumerate}
\end{theorem}
\begin{proof}
	\see{48}
	By showing $3. \implies 2. \implies 1. \implies 3.$
\end{proof}

\begin{theorem}[Semi-decidability and decidability of complete theories]
	Let $\mathfrak{T}$ be a complete theory.
	Then, $\mathfrak{T}$ is recursively enumerable iff it is decidable.
\end{theorem}
\begin{proof}
	"$\impliedby$": Always holds.\\
	"$\implies$":
	For $\phi(\overline{x})$ we know by Lemma \ref{lem:universal closure in complete theories} that
	\[
	\phi \in \mathfrak{T} \iff \psi \defeq \forall \overline{x}.\phi \in \mathfrak{T}
	.\]
	This can be decided using a recursive enumeration $\theta_1, \theta_2, \ldots$ of $\mathfrak{T}$,
	by stopping as soon as $\psi$ or $\neg \psi$ is encountered.
	This procedure terminates since $\psi$ is a sentence and $\mathfrak{T}$ is complete.
\end{proof}

\newpage
\subsection{Axiomatizability and decidability}
\medskip
\begin{definition}[Axiomatizable theories]
	A theory $\mathfrak{T}$ is called \textit{axiomatizable} if it is the logical closure of a decidable (sub)set, i.e.:
	\[
		\mathfrak{T} = \left\{ \phi \mid \mathfrak{F} \Vdash \phi\right\} = \func{Cl}(\mathfrak{F})
	.\]
	$\mathfrak{F}$ is called the \textit{axiomatization} of $\mathfrak{T}$.
	If $\mathfrak{F}$ is finite, $\mathfrak{T}$ is called \textit{finitely axiomatizable}.
\end{definition}
\begin{note}
	If $\mathfrak{F} = \left\{ \phi_1, \ldots, \phi_n \right\}$ is a finite axiomatization of $\mathfrak{T}$,
	then $\left\{ \bigwedge_{i=1}^{n} \phi_i \right\}$ is an axiomatization of $\mathfrak{T}$ with only one element.
\end{note}
