\chapter{Second order logic}
\lecture{16}{17.06.2021}{}
With second order logic (SOL) we now discuss a powerful extension of FOL,
where one is allowed to quantify over predicates and functions.
Since FOL already contains undecidable problems and SOL is a true superset of FOL,
we cannot expect these problems to be decidable for SOL.

\section{Syntax and semantics of SOL}
We assume a fixed vocabulary $\var{Voc} = \left( \var{Pred}, \var{Func} \right)$
and need to deal with three kinds of variables:
\begin{itemize}
	\item FO-variables $x, y, z, \ldots \in \var{Var}$: as in FOL elements of the domain
	\item predicate variables $X, Y, Z, \ldots \in \var{PVar}$: predicates over the domain
	\item function variables $F, G, H, \ldots \in \var{FVar}$ : functions over the domain
\end{itemize}
Predicate variables and function variables are often called second-order variables.
Like predicate or function symbols they are assumed to be classified by their arity:
\[
	\var{PVar} = \biguplus_{i \geq 0} \var{PVar}_{i},\qquad \var{FVar} = \biguplus_{i \geq 1} \var{FVar}_i
\]
where $\var{PVar}_n$ and $\var{FVar}_m$ are predicates of arity $n$ and $m$, respectively.
We skip $\var{FVar}_0$, because its elements would be equivalent to first-order variables.
Obviously, $\var{Var}, \var{PVar}, \var{FVar}$ and $\var{Voc}$ all need to be pairwise disjoint.

\subsection{Syntax of SOL}
