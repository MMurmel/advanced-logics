\newpage
\lecture{3}{22.04.21}{}[1cm]
\begin{definition}[Sondness]
	A Hilbert proof system $\mathfrak{D}$ is called \textit{sound}, if for each formula-set $\mathfrak{F}$
	and each formula $\phi$ the following holds:
	\[
		\mathfrak{F} \vdash_{\mathfrak{D}} \phi \text{ implies } \mathfrak{F} \Vdash \phi
	.\]
\end{definition}
\begin{lemma}[Criterion for soundness]
	$\mathfrak{D}$ is sound if and only if the following hold:
	\begin{itemize}
		\item all instances of the axioms in $\mathfrak{D}$ are valid and
		\item whenever $\left( \phi_1, \ldots, \phi_n, \phi \right)$ is an instance of a proof rule in $\mathfrak{D}$,
			then $\left\{ \phi_1, \ldots, \phi_n \right\} \Vdash \phi$.
	\end{itemize}
\end{lemma}

\begin{definition}[Completeness and weak completeness]
	A Hilbert proof system $\mathfrak{D}$ is called \textit{complete}, if for each formula-set $\mathfrak{F}$
	and each formula $\phi$ the following holds:
	\[
		\mathfrak{F} \Vdash \phi \text{ implies } \mathfrak{F} \vdash_{\mathfrak{D}} \phi
	.\]
	\textit{Weak completeness} is the property, that all valid formula are $\mathfrak{D}$-provable, i.e.:
	\[
		\Vdash \phi \text{ implies } \vdash_{\mathfrak{D}} \phi
	.\]
\end{definition}

Soundness and completeness are really only two sides of this implication, 
hence for sound and complete $\mathfrak{D}$ we have $\mathfrak{F} \vdash_{\mathfrak{D}} \phi \iff \mathfrak{F} \Vdash \phi$.

\begin{theorem}[Gödel's completeness theorem]
	\see{20-22}
	There exists a Hilbert proof system that is sound and complete for FOL.
\end{theorem}
In the lecture modus ponens and eight axioms are used to build the sound and complete proof system $\mathfrak{D}_{\var{PL}}$
for propositional logic.

For FOL, the proof system $\mathfrak{D}_{\var{FOL}}$ is shown, consisting of modus ponens and the following axioms:
\begin{itemize}
	\item (Q0) the set of all generalizations of propositional tautologies
	\item (Q1) $\forall x.\Phi \to \Phi[x/t]$ if $x$ can be replaced with $t$ in $\Phi$ 
	\item (Q2) $\Phi \to \forall x.\Phi$ if $x \notin \func{Free}(\Phi)$
	\item (Q3) $\forall x.(\Phi \to \Psi) \to \left( \forall x.\Phi \to \forall x.\Psi \right)$
\end{itemize}

\begin{theorem}[Soundness and completeness of $\mathfrak{D}_{\var{FOL}}$]
	\see{23f.}
	$\mathfrak{D}_{\var{FOL}}$ is sound and complete, i.e.,
	if $\mathfrak{F}$ is a set of FOL-formulas and $\phi$ a FOL-formula without equality then:
	\[
		\mathfrak{F} \Vdash \phi \iff \mathfrak{F} \vdash_{\mathfrak{D}_{\var{FOL}}} \phi
	.\]
\end{theorem}

A property useful in the completeness proof for $\mathfrak{D}_{\var{FOL}}$ is the following deduction property.
\begin{lemma}[Deduction property]
	Let $\mathfrak{D}$ be a Hilbert proof system that is complete for propositional logic
	and uses the modus ponens, but no other proof rule of arity one or more.
	Then, for all sets $\mathfrak{F}$ of formula and all formulas $\phi$:
	\[
		\mathfrak{F} \vdash_{\mathfrak{D}} \phi \to \psi \iff \mathfrak{F} \cup \left\{ \phi \right\} \vdash_{\mathfrak{D}} \psi
	.\]
\end{lemma}
\begin{note}
	Whenever $\mathfrak{D}$ is a Hilbert proof system that enjoys the deduction property,
	then the modus ponens is a derivable proof rule.
	\see{25f.}
\end{note}

Deductive calculi form the bridge between satisfiability and the (syntactic) notion of \textit{consistency}.
\begin{lemma}[$\mathfrak{D}$-(in)consistency]
	Let $\mathfrak{D}$ be a Hilbert proof system that is complete for propositional logic
	and enjoys the deduction property.
	Then, for all formula sets $\mathfrak{F}$ the following are equivalent:
	\begin{itemize}
		\item $\mathfrak{F} \vdash_{\mathfrak{D}} \var{false}$ 
		\item There exists a formula $\psi$ such that $\mathfrak{F} \vdash_{\mathfrak{D}} \psi$ and $\mathfrak{F} \vdash_{\mathfrak{D}} \neg \psi$
		\item $\mathfrak{F} \vdash_{\mathfrak{D}} \phi$ for all formulas $\phi$
	\end{itemize}
	If one of these conditions hold, $\mathfrak{D}$ is called \textit{inconsistent}, otherwise \textit{consistent}.
\end{lemma}

\begin{theorem}[Completeness and consistency]
	Let $\mathfrak{D}$ be a Hilbert proof system that is complete for propositional logic
	and enjoys the deduction property.
	Then the following statements are equivalent:
	\begin{itemize}
		\item $\mathfrak{D}$ is complete, i.e., $\mathfrak{F} \Vdash \phi$ implies $\mathfrak{F} \vdash_{\mathfrak{D}} \phi$ 
		\item $\mathfrak{D}$-consistency implies satisfiability, i.e., each $\mathfrak{D}$-consistent formula set is satisfiable.
	\end{itemize}
\end{theorem}
\begin{note}
	This does not only apply to FOL, but for every logic and deductive calculus that meets the deduction property and is complete for propositional logic.
	For sound and complete proof systems, the notions $\mathfrak{D}$-consistency and satisfiability coincide, i.e.\ :
	\[
		\mathfrak{D} \text{ is sound and complete} \iff \begin{cases}
			\text{for all formula sets $\mathfrak{F}$:} \\
			\mathfrak{F} \text{ is $\mathfrak{D}$-consistent} \iff \mathfrak{F} \text{ is satisfiable}
		\end{cases}
	\]
\end{note}
