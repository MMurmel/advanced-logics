\subsubsection{Isomorphism and homomorphism}
\lecture{2}{16.04.2021}{}
Two structures $\mathcal{A}, \mathcal{B}$ for the same vocabulary $\var{Voc}$ are called \textit{isomorphic} if
there exists a bijection $h : \Delta^\mathcal{A} \to \Delta^\mathcal{B}$ such that
\begin{itemize}
	\item $P^\mathcal{B} = \left\{ (h(a_1), \ldots, h(a_n)) \mid (a_1, \ldots, a_n) \in P^\mathcal{A} \right\}$ for all $P \in \var{Pred}_n$,
	\item $f^\mathcal{B}(h(a_1), \ldots, h(a_m)) = h(f^\mathcal{A}(a_1, \ldots, a_m))$ for all $a_1,\ldots,a_m \in \Delta^\mathcal{A}$ and $f \in \var{Func}_m$.
\end{itemize}
In this case, $h$ is called an \textit{isomorphism} from $\mathcal{A}$ to $\mathcal{B}$ and we have:
\[
	(\mathcal{A}, \mathcal{V}) \vDash \phi \iff (\mathcal{B}, h \circ \mathcal{V}) \vDash \phi
\]
for all formulas $\phi$.
The satisfaction relation $\vDash$ does not distinguish between isomorphic structures, i.e.\ structures that are the same up to renaming of elements.
\see{p. 11}

A \textit{homomorphism} from $\mathcal{A}$ to $\mathcal{B}$ is a function $h : \Delta^\mathcal{A} \to \Delta^\mathcal{B}$ such that
\begin{itemize}
	\item $(a_1, \ldots, a_n) \in P^\mathcal{A} \iff (h(a_1), \ldots, h(a_n)) \in P^\mathcal{B}$ for all $P \in \var{Pred}_n$ and $a_1, \ldots, a_n \in \Delta^\mathcal{A}$ 
	\item $f^\mathcal{B}(h(a_1), \ldots, h(a_m)) = h(f^\mathcal{A}(a_1, \ldots, a_m))$ for all $a_1,\ldots,a_m \in \Delta^\mathcal{A}$ and $f \in \var{Func}_m$.
\end{itemize}
Homomorphism in general do however only preserve the truth values of quantifier-free FOL-formulas without equality.
For surjective homomorphism we gain:
\[
	(\mathcal{A}, \mathcal{V}) \vDash \phi \iff (\mathcal{B}, h \circ \mathcal{V}) \vDash \phi
\]
for formulas $\phi$ without equality.

\subsection{Positive normal forms (PNF)}
In \textit{positive normal form} (PNF), also called \textit{negation normal form},
negations may only appear in front of atomic formulas $P(t_1, \ldots, t_n)$ and $t_1 = t_2$.
We therefore also need $\lor$ and $\exists$ in the syntax.
We also define $\var{false} \coloneqq \neg \var{tue}$ and $t_1 \neq t_2 \coloneqq \neg (t_1 = t_2)$.

Every FOL-formula $\phi$ can be transformed into an equivalent PNF-formula.

\subsection{Prenex and Skolem form}
\subsubsection{Prenex form}
A FOL-formula is in \textit{prenex form} if it is of type $Q_1 x_1 \ldots Q_n x_n.\psi$, where $Q_i \in \left\{ \forall, \exists \right\}$
and $\psi$ is a quantifier free formula, called the \textit{matrix}.
Every FOL-formula $\phi$ has an equivalent formula in prenex form obtained by:
\begin{enumerate}
	\item constructing an equivalent PNF-formula $\phi_{\var{PNF}}$ 
	\item creating an equivalent formula $\phi_{\var{PNF}}'$ such that no variable is bound twice and no variable occurs both bound and free
	\item transforming $\phi_{\var{PNF}}'$ into an equivalent prenex formula $\phi_{\var{prenex}}$ by moving all quantifiers to the left.
\end{enumerate}
Step 2 can be accomplished with the concept of bound renaming.
To perform step 3 the following equivalences are needed:
\begin{lemma}
Let $\phi, \psi$ be FOL-formulas, $x \in \var{Var}$ and $x$ does not appear in $\psi$.
Then the following hold:
\begin{align*}
	(\forall x.\phi) \land \psi &\equiv \forall x.(\phi \land \psi) \\
	(\exists x. \phi) \land \psi &\equiv \exists x.(\phi \land \psi)\\
	(\forall x.\phi) \lor \psi &\equiv \forall x.(\phi \lor \psi) \\
	(\exists x. \phi) \lor \psi &\equiv \exists x.(\phi \lor \psi)
\end{align*}
\end{lemma}

\subsubsection{Skolem form}
A FOL-formula is in \textit{skolem form} if it is in prenex form and all quantifiers are universal,
i.e.\ $Q_1 = \ldots = Q_n = \forall$.
Every FOL-formula $\phi$ has an equisatisfiable formula in skolem form obtained by:
\begin{enumerate}
	\item transforming $\phi$ into an equivalent prenex FOL-formula $\phi_{\var{prenex}}$ 
	\item replacing all existential quantifiers.
		Consider the left most existential quantifier in $\phi$:
		\[
		\phi = \forall x_1 \ldots \forall x_k \exists y.\psi
		.\]
		We obtain an equisatisfiable formula $\phi'$ by:
		\[
			\phi' = \forall x_1 \ldots \forall x_k.\psi[y/f(x_1, \ldots, x_k)]
		,\]
		where $f$ is a new $k$-ary function symbol.
		We can then recursively apply this transformation until there is no existential quantifier left and thereby obtain $\phi_{\var{skolem}}$.
\end{enumerate}

\subsection{FOL without function symbols}
\see{p. 13-15}
FOL with purely relational vocabularies is as expressive as full FOL.
Every FOL-formula has an equivalent FOL-formula over a purely relational vocabulary in the following sense:
Given a formula $\phi$ over a vocabulary $\var{Voc}$, there is a formula $\phi_{rel}'$ over a purely relational vocabulary $\var{Voc}_{\var{rel}}$,
such that $\phi$ is satisfiable over some structure with domain $\Delta$ iff so is $\phi_{\var{rel}}'$.
Furthermore, $\phi$ and $\phi_{\var{rel}}'$ have the same asymptotic (word-) lengths and there is an algorithm to compute $\phi_{\var{rel}}'$ from $\phi$.
