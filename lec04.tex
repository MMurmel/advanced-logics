\lecture{4}{23.04.2021}{}
Hilbert proof systems can even be defined in a more general way.
When done like in the lecture, they are also applicable to FOL over arbitrary (possibly uncountable) vocabularies and variable-sets. 
\see{29}
By this more general notion, one obtains the fact, that there are sound and complete Hilbert proof systems,
even for FOL over arbitrary vocabularies and variable-sets.
Therefore, all properties derived from Gödel's completeness theorem for FOL hold for arbitrary vocabularies;
except for the consideration of the following Lemma \ref{lem:semi decidability FOL tautologies}.

\begin{lemma}[Semi-decidability of FOL-tautologies]\label{lem:semi decidability FOL tautologies}
	The set of all valid FOL-formulas (over some recursively enumerable sets $\var{Voc}$ and $\var{Var}$) is recursively enumerable.
	\see{29f.}
\end{lemma}
\begin{note}
	Since the existence of a recursive enumeration is equivalent to semi-decidability,
	the set of all valid FOL-formulas is also semi-decidable,
	i.e.\ there exists a algorithm that, when run on a FOL formula $\phi$ as input,
	will terminate and answer "yes" if $\phi$ is a tautology.
	In the case that $\phi$ is not a tautology the algorithm may halt and answer "no"
	or it may not terminate.
	
	Furthermore, in the proof of this lemma, no property specific to FOL is used.
	The mere existence of sound and weakly complete deductive calculi for a logic is enough
	to imply that the set of all tautologies of that logic is recursively enumerable.
\end{note}

\newpage

\section{Undecidability of FOL}
Even though, the set of all valid FOL-formulas is recursively enumerable, and hence semi-decidable,
it is not decidable.
In this section, we will show the undecidability of important decision problems for FOL.
We will focus on an underlying vocabulary $\var{Voc}_{\omega}$ of countably many predicate and function symbols of arbitrary arity.
\see{30}
The details involved with this assumption will not be considered, as they are fairly technical.

\begin{theorem}[Undecidability of FOL]
	The following problems for FOL-sentences over $\var{Voc}_{\omega}$ are undecidable:
	\begin{itemize}
		\item validity problem: "is $\phi$ valid?"
		\item satisfiability problem: "is $\phi$ satisfiable?"
		\item equivalence problem: "does $\phi \equiv \psi$ hold?"
		\item consequence problem: "does $\phi \Vdash \psi$ hold?"
	\end{itemize}
\end{theorem}
\begin{proof}
	At first, we observe that undecidability of one of these problems also implies undecidability of the others:
	\[
	\phi \text{ is valid } \iff  \neg \phi \text{ is not satisfiable } \iff \var{true} \equiv \phi \iff \var{true} \Vdash \phi
	.\]
	It remains to show, that one of them is indeed undecidable.
	\see{31ff.}
	In the script this is shown by a reduction from the \textit{Post's correspondence problem} (PCP),
	which is known to be undecidable.
	\qedhere
\end{proof}

\subsection{Tragkatenbrots Theorem}
One could have the idea, that the undecidability of FOL is a result of the fact that models can have infinite domains.
However, nothing could be further from the truth.
We will show that even the finitary validity problem "does a given FOL-sentence hold for all finite structures?" is undecidable.

\begin{theorem}[Finite model property violated by FOL]
	There exist satisfiable FOL-sentences that do not have a finite model,
	and FOL-sentences that hold for all finite structures, but are not valid.
\end{theorem}
\begin{proof}
	Let $P \in \var{Pred}_2$. We define $\phi \defeq \phi_1 \land \phi_2 \land \neg \phi_3$ where:
	\begin{align*}
		\phi_1 &\defeq \forall x \exists y. P(x,y) \\
		\phi_2 &\defeq \forall x \forall y \forall z. (P(x,y) \land P(y, z) \to P(x,z)) \\
		\phi_3 &\defeq \exists x.P(x,x)
	\end{align*}
	One can easily show that every model of $\phi$ is infinite.
	
	Furthermore, $\phi' \defeq \phi_1 \land \phi_2 \to \phi_3$ can be shown to hold for all finite structures.
	However, it is not valid.
\end{proof}
Thus, finitary satisfiability is weaker that the standard notion of satisfiability,
and the set of valid FOL-sentences is a true subset of the set of FOL-sentences that are true for all finite structures.

The following still focuses on the vocabulary $\var{Voc}_\omega$ and an infinite set $\var{Var}$ of variables.
\begin{notation}
	We use special names for the sets discussed earlier:
	\begin{itemize}
		\item $\var{FOL-SAT}$: the set of FOL-sentences that are satisfiable
		\item $\var{FOL-VALID}$: the set of valid FOL-sentences 
		\item $\var{FOL-SAT-FIN}$: the set of FOL-sentences having at least one finite model
		\item $\var{FOL-VALID-FIN}$: the set of FOL-sentences holding for all finite structures
	\end{itemize}
\end{notation}
The following results are shown in the lecture in great detail:
\see{35-42}
\begin{figure}[H]
	\centering
	\includegraphics[width=\textwidth]{./figures/satisfaction_inclusions.png}
\end{figure}
